\vspace{-0.4cm}\section{RELATED WORK}
Aiming at improving the performance of image search, a series of methods have been proposed to capture human cognition , including query log based methods \cite{click1,  click_bing}, query analysis based methods \cite{query_ana1,mingdong}, relevance feedback based methods \cite{rf, attribute_feedback}, etc. In query log based methods, user click data in image search engines are used to estimate user intention. However, if a query contains less training images, the performance will not be very good. Besides, the image with low rank will not be easily seen by others. Query analysis based methods usually use the techniques in IR, such as query suggestion to capture different aspects of user intention. Zha \emph{et al.} proposed a visual query suggestion approach \cite{visual_suggestion} to suggest more detailed queries for ambiguous queries. In these methods, an important assumption is that visually similar images should have similar user intentions, which is not always tenable. Relevance feedback is another effective way to collect the cognition information by collecting users' feedback. However, the complex operation of feedback may sometimes reduce the user experience.

With the development of social media, a series of social-sensed image search and recommendation approaches have been proposed\cite{cui2014}. The social factors, such as image tags, users, interest groups are considered to replace the original manually labeled data. Image tagging methods \cite{tag1} by user annotation show their significant improvements in bridging the semantic gap. Liu \emph{et al.} proposed an image reranking method \cite{social_visual} that considers both visual factor\cite{zheng2013visual,zhiwang} and social factor. In this work, interest group in Flickr is utilized to evaluate the image similarity in user intention level. The research indicates that the interest groups can help understanding user intention in image reranking. However, this work is based on the images in Flickr, which cannot be well generalized to the ordinary Web images without social information such as interest groups.

Image distance metric plays an important role in many machine learning problems. Traditional metric learning researches usually aim at learning metric from labeled examples. The methods can be categorized into supervised ones \cite{super_metric1} and semi-supervised ones \cite{semi_super_metric}. In supervised metric learning, labels of images are complete, such as the categories of the images. Kilian \emph{et al.} proposed a method named LMNN \cite{lmnn}, which aims at reducing the margin of nearest neighbors. In semi-supervised metric learning, we do not have all the labels but only know some pairs of images are similar and some pairs are dissimilar. Thus, these methods aim at reducing the distance among the similar set and enlarging the distance among the dissimilar set. In our work, we do not have any labeled images but the images with social behavioral information. Although the social similarity can be evaluated by the social information, its reliability is not guaranteed because the social data are very noisy and uncertain. In addition, social similarity is a wholly new dimension to evaluate image similarity and it is very sparse. Thus visual distance needs to be maintained when an image does not have a socially similar neighbor.



